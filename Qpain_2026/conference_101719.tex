\documentclass[conference]{IEEEtran}
\IEEEoverridecommandlockouts
\renewcommand{\thesection}{\Roman{section}}
\usepackage{booktabs}
\usepackage{fancyhdr}
\usepackage{algorithm}
\usepackage[usenames, dvipsnames]{xcolor}
\usepackage{algorithmic}
\usepackage{colortbl}
\usepackage{caption}
\usepackage{graphicx}
\usepackage{array}
\usepackage{adjustbox}
\usepackage{hyperref,graphicx,color,float}
\bibliographystyle{unsrt}
\definecolor{myblue}{RGB}{10, 150, 200}

\usepackage{cite}
\usepackage{multirow}
\usepackage{amsmath,amssymb,amsfonts}
\usepackage{algorithmic}
\usepackage{graphicx}
\usepackage{textcomp}
\usepackage{comment}
\definecolor{highlightColor}{HTML}{E6FFE6}

\usepackage{xcolor}
\def\BibTeX{{\rm B\kern-.05em{\sc i\kern-.025em b}\kern-.08em
    T\kern-.1667em\lower.7ex\hbox{E}\kern-.125emX}}
    
\fancypagestyle{firstpage}{
  \fancyhf{} % Clear all header and footer fields
  \fancyhead[L]{\small 2026 IEEE 2nd International Conference on Quantum Photonics, Artificial Intelligence, and Networking (QPAIN)
 \\ 16 – 18 April 2026, Chittagong, Bangladesh}
  \fancyfoot[L]{\small 979-8-3315-4990-9/26/\$31.00 \copyright2026 IEEE} % Left-align the footer
  \renewcommand{\headrulewidth}{0pt} % Remove the header rule
  \renewcommand{\footrulewidth}{0pt} % Remove the footer rule
}

\pagestyle{plain} % Apply plain style to other pages

\title{Paper Title}
\begin{document}
\maketitle
\thispagestyle{firstpage} % Apply the footer only on the first page

\begin{abstract}
This is a placeholder abstract. Briefly summarize the scope and contributions of your paper here. For example, "This template provides a reusable structure for IEEE conference papers, including sections for abstract, introduction, methodology, results, tables, figures, and citations."
\end{abstract}

\begin{IEEEkeywords}
Placeholder, IEEE Template, Example Paper, Structure, Keywords
\end{IEEEkeywords}

\section{Introduction}
The introduction goes here. State the research problem, motivation, and brief summary of related work. Cite prior work as needed~\cite{b2, b3}. 
This template helps you quickly create a professional-looking IEEE paper.

\section{Literate Review}
A literature review is a comprehensive analysis of published research on a specific topic. It summarizes, analyzes, and synthesizes existing knowledge, providing an overview of current understanding and identifying gaps in the research. Essentially, it helps researchers and readers understand the current state of knowledge in a particular field. 

\section{Methodology}
Describe your methods. Explain how you collect or generate data, the algorithms or analysis you use, and any theoretical framework. Include a sample table (Table~\ref{tab:sampletable}) and a placeholder figure (Figure~\ref{fig:samplefig}) as examples.

\subsection{Data Collection}
Explain your data sources or experimental setup here.

\subsection{Analysis}
Describe any models, tools, or analyses applied. For example, "A baseline model was compared with advanced algorithms as described by Author et al.~\cite{b4}."

\begin{table}[htbp]
\caption{Sample Placeholder Table}
\label{tab:sampletable}
\centering
\begin{tabular}{|c|c|c|}
\hline
\textbf{Method} & \textbf{Accuracy (\%)} & \textbf{Reference} \\
\hline
Method A & 90.2 & \cite{ref4} \\
Method B & 92.5 & \cite{ref5} \\
Method C & 88.9 & \cite{ref6} \\
\hline
\end{tabular}
\end{table}

\begin{figure}[htbp]
\centering
\includegraphics[width=0.3\textwidth]{placeholder_image.jpg}
\caption{Sample Placeholder Image or Diagram}
\label{fig:samplefig}
\end{figure}

\section{Results and Discussions}
Discuss your results. Use tables, charts, and figures as needed. Refer to findings with citation~\cite{b6}.

\section{Discussion}
Interpret your results. Highlight implications, limitations, and possible improvements. Mention relevant related work for context.

\section{Conclusion}
Summarize the paper, key results, and possible future work. This section wraps up the main contributions and suggests areas for additional research.


















\bibliographystyle{IEEEtran}
\bibliography{Ref}
\end{document}
